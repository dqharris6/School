\documentclass[12pt]{report}
\usepackage{amsmath,amsthm,latexsym,paralist}
\usepackage[document]{ragged2e}

\theoremstyle{definition}
\newtheorem{problem}{Problem}

\begin{document}

\begin{problem} 	R-5.3 (5 points) 		
\end{problem}
				The size of the stack is 25 - 10 + 3 = 18.
				
\begin{problem} 	R-5.4 (5 points)		
\end{problem}
				The current value of top() is 18.

\begin{problem} 	R-5.5 (5 points)		
\end{problem}
				The output is as follows: 3, 8, 2, 1, 6, 7, 4, 9.
				
\begin{problem} 	R-5.9 (5 points)		
\end{problem}
				The output is as follows: 5, 3, 2, 8, 9, 1, 7, 6.

\begin{problem} 	R-5.10 (5 points)		
\end{problem}
				The output is as follows: 5, 9, 3, 3, 7, 7.

\begin{problem} 	C-5.8 (10 points)		
\end{problem}
				First, you'll want to tokenize the expression into just its operators and numbers. Then, iterate 				through the tokens in order; if its a number then push it onto the stack. If not, then pop 2 					operands off the stack, and push the result onto the stack. Finally, the answer is the only 					number on the stack that evaluates the whole expression.

\goodbreak
\end{document}